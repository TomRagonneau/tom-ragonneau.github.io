% English CV of Tom M. Ragonneau (05/05/2020).
\PassOptionsToPackage{dvipsnames}{xcolor}
\documentclass[10pt,a4paper,academicons]{altacv}
%% AltaCV uses the fontawesome and academicon fonts and packages.
%% See texdoc.net/pkg/fontawecome and http://texdoc.net/pkg/academicons for full list of symbols.


% Change the page layout if you need to
\geometry{left=1cm,right=9cm,marginparwidth=6.8cm,marginparsep=1.2cm,top=1.25cm,bottom=1.25cm,footskip=2\baselineskip}

\usepackage[T1]{fontenc}
\usepackage[utf8]{inputenc}
\usepackage[default]{lato}
\setmainfont{Lato}

\definecolor{Mulberry}{HTML}{0F4C81}
\definecolor{SlateGrey}{HTML}{2E2E2E}
\definecolor{LightGrey}{HTML}{666666}
\colorlet{heading}{Mulberry!70!black}
\colorlet{accent}{Mulberry}
\colorlet{emphasis}{SlateGrey}
\colorlet{body}{LightGrey}

\renewcommand{\itemmarker}{{\small\textbullet}}
\renewcommand{\ratingmarker}{\faCircle}

\usepackage[colorlinks]{hyperref}

\begin{document}

\name{TOM M. RAGONNEAU}
\tagline{\'{E}tudiant doctoral, Mathématiques Computationnelles et Optimisation}
% \photo{2.8cm}{Globe_High}
\personalinfo{%
    % Not all of these are required!
    \email{tom.ragonneau@connect.polyu.hk}
    \mailaddress{P115, Mong Man Wai Bldg., The Hong Kong Polytechnic University}
    \location{Hong Kong}
    \homepage{www.tom-ragonneau.co}
%    \twitter{@twitterhandle}
%    \linkedin{linkedin.com/in/yash-raj-579b96156/}
    \github{github.com/TomRagonneau}
    \orcid{orcid.org/0000-0003-2717-2876}
}

\begin{fullwidth}
\makecvheader
\end{fullwidth}

\cvsection[page1sidebar]{\'{E}ducation}

\cvevent{\'{E}tudiant doctoral en Math\'{e}matiques Computationnelles}{Universit\'{e} Polytechnique de Hong Kong}{Sep. 2019 -- Aujourd'hui}{Hong Kong}
\begin{itemize}
    \item D\'{e}partement de Math\'{e}matiques Appliqu\'{e}es.
    \item Supervis\'{e} par Dr. Zaikun Zhang et Prof. Xiaojun Chen.
    \item Soutenu par le Conseil de Subvention de Recherche (RGC) de Hong Kong, sous le Projet de Subvention Doctorale de Hong Kong (HKPFS).
\end{itemize}

\divider

\cvevent{Dipl\^{o}me M.Sc. en Calcul Scientifique}{Toulouse INP, E.N.S.E.E.I.H.T.}{Sep. 2018 - Jui. 2019}{Toulouse, France}
\begin{itemize}
    \item Diplom\'{e} en Performance des Softwares, des Médias et des Calculs Scientifiques (PSMSC).
    \item GPA: 4.0
\end{itemize}

\divider

\cvevent{Dipl\^{o}me d'ing\'{e}nieur en HPC et Big Data}{Toulouse INP, E.N.S.E.E.I.H.T.}{Sep. 2016 - Jui. 2019}{Toulouse, France}
\begin{itemize}
    \item Département d'Informatique et de Mathématiques Appliquées.
    \item Sp\'{e}cialis\'{e} en optimisation, HPC et machine learning.
    \item GPA: 3.9
\end{itemize}

\cvsection{Exp\'{e}rience professionnelle}

\cvevent{Assistant de Recherche}{Universit\'{e} Polytechnique de Hong Kong}{Mar 2019 -- Sep. 2019}{Hong Kong}
\begin{itemize}
    \item D\'{e}partement de Math\'{e}matiques Appliqu\'{e}es.
    \item Stage de derni\`{e}re ann\'{e}e du dipl\^{o}me d'ing\'{e}nieur.
\end{itemize}

\divider

\cvevent{Recherche en Machine Learning}{Toulouse INP, E.N.S.E.E.I.H.T. \& ALTRAN}{Jan. 2019 -- Mar. 2019}{Toulouse, France}
\begin{itemize}
    \item Estimation de la bathym\'{e}trie des littoraux par deep learning.
    \item Projet de derni\`{e}re ann\'{e}e du dipl\^{o}me d'ing\'{e}nieur.
\end{itemize}

\divider

\cvevent{Ing\'{e}nieurie en Machine Learning}{Axians Cloud Builder}{Jui. 2018 -- Sep. 2018}{Toulouse, France}
\begin{itemize}
    \item Pr\'{e}diction de la charge d'un cluster HPC (Centre National d'\'{E}tudes Spatiales) manag\'{e} par GPFS, via des outils de machine learning.
    \item Stage de deuxi\`{e}me ann\'{e}e du dipl\^{o}me d'ing\'{e}nieur.
\end{itemize}

\end{document}
